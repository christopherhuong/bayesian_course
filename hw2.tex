% Options for packages loaded elsewhere
\PassOptionsToPackage{unicode}{hyperref}
\PassOptionsToPackage{hyphens}{url}
%
\documentclass[
]{article}
\usepackage{amsmath,amssymb}
\usepackage{lmodern}
\usepackage{iftex}
\ifPDFTeX
  \usepackage[T1]{fontenc}
  \usepackage[utf8]{inputenc}
  \usepackage{textcomp} % provide euro and other symbols
\else % if luatex or xetex
  \usepackage{unicode-math}
  \defaultfontfeatures{Scale=MatchLowercase}
  \defaultfontfeatures[\rmfamily]{Ligatures=TeX,Scale=1}
\fi
% Use upquote if available, for straight quotes in verbatim environments
\IfFileExists{upquote.sty}{\usepackage{upquote}}{}
\IfFileExists{microtype.sty}{% use microtype if available
  \usepackage[]{microtype}
  \UseMicrotypeSet[protrusion]{basicmath} % disable protrusion for tt fonts
}{}
\makeatletter
\@ifundefined{KOMAClassName}{% if non-KOMA class
  \IfFileExists{parskip.sty}{%
    \usepackage{parskip}
  }{% else
    \setlength{\parindent}{0pt}
    \setlength{\parskip}{6pt plus 2pt minus 1pt}}
}{% if KOMA class
  \KOMAoptions{parskip=half}}
\makeatother
\usepackage{xcolor}
\usepackage[margin=1in]{geometry}
\usepackage{color}
\usepackage{fancyvrb}
\newcommand{\VerbBar}{|}
\newcommand{\VERB}{\Verb[commandchars=\\\{\}]}
\DefineVerbatimEnvironment{Highlighting}{Verbatim}{commandchars=\\\{\}}
% Add ',fontsize=\small' for more characters per line
\usepackage{framed}
\definecolor{shadecolor}{RGB}{248,248,248}
\newenvironment{Shaded}{\begin{snugshade}}{\end{snugshade}}
\newcommand{\AlertTok}[1]{\textcolor[rgb]{0.94,0.16,0.16}{#1}}
\newcommand{\AnnotationTok}[1]{\textcolor[rgb]{0.56,0.35,0.01}{\textbf{\textit{#1}}}}
\newcommand{\AttributeTok}[1]{\textcolor[rgb]{0.77,0.63,0.00}{#1}}
\newcommand{\BaseNTok}[1]{\textcolor[rgb]{0.00,0.00,0.81}{#1}}
\newcommand{\BuiltInTok}[1]{#1}
\newcommand{\CharTok}[1]{\textcolor[rgb]{0.31,0.60,0.02}{#1}}
\newcommand{\CommentTok}[1]{\textcolor[rgb]{0.56,0.35,0.01}{\textit{#1}}}
\newcommand{\CommentVarTok}[1]{\textcolor[rgb]{0.56,0.35,0.01}{\textbf{\textit{#1}}}}
\newcommand{\ConstantTok}[1]{\textcolor[rgb]{0.00,0.00,0.00}{#1}}
\newcommand{\ControlFlowTok}[1]{\textcolor[rgb]{0.13,0.29,0.53}{\textbf{#1}}}
\newcommand{\DataTypeTok}[1]{\textcolor[rgb]{0.13,0.29,0.53}{#1}}
\newcommand{\DecValTok}[1]{\textcolor[rgb]{0.00,0.00,0.81}{#1}}
\newcommand{\DocumentationTok}[1]{\textcolor[rgb]{0.56,0.35,0.01}{\textbf{\textit{#1}}}}
\newcommand{\ErrorTok}[1]{\textcolor[rgb]{0.64,0.00,0.00}{\textbf{#1}}}
\newcommand{\ExtensionTok}[1]{#1}
\newcommand{\FloatTok}[1]{\textcolor[rgb]{0.00,0.00,0.81}{#1}}
\newcommand{\FunctionTok}[1]{\textcolor[rgb]{0.00,0.00,0.00}{#1}}
\newcommand{\ImportTok}[1]{#1}
\newcommand{\InformationTok}[1]{\textcolor[rgb]{0.56,0.35,0.01}{\textbf{\textit{#1}}}}
\newcommand{\KeywordTok}[1]{\textcolor[rgb]{0.13,0.29,0.53}{\textbf{#1}}}
\newcommand{\NormalTok}[1]{#1}
\newcommand{\OperatorTok}[1]{\textcolor[rgb]{0.81,0.36,0.00}{\textbf{#1}}}
\newcommand{\OtherTok}[1]{\textcolor[rgb]{0.56,0.35,0.01}{#1}}
\newcommand{\PreprocessorTok}[1]{\textcolor[rgb]{0.56,0.35,0.01}{\textit{#1}}}
\newcommand{\RegionMarkerTok}[1]{#1}
\newcommand{\SpecialCharTok}[1]{\textcolor[rgb]{0.00,0.00,0.00}{#1}}
\newcommand{\SpecialStringTok}[1]{\textcolor[rgb]{0.31,0.60,0.02}{#1}}
\newcommand{\StringTok}[1]{\textcolor[rgb]{0.31,0.60,0.02}{#1}}
\newcommand{\VariableTok}[1]{\textcolor[rgb]{0.00,0.00,0.00}{#1}}
\newcommand{\VerbatimStringTok}[1]{\textcolor[rgb]{0.31,0.60,0.02}{#1}}
\newcommand{\WarningTok}[1]{\textcolor[rgb]{0.56,0.35,0.01}{\textbf{\textit{#1}}}}
\usepackage{graphicx}
\makeatletter
\def\maxwidth{\ifdim\Gin@nat@width>\linewidth\linewidth\else\Gin@nat@width\fi}
\def\maxheight{\ifdim\Gin@nat@height>\textheight\textheight\else\Gin@nat@height\fi}
\makeatother
% Scale images if necessary, so that they will not overflow the page
% margins by default, and it is still possible to overwrite the defaults
% using explicit options in \includegraphics[width, height, ...]{}
\setkeys{Gin}{width=\maxwidth,height=\maxheight,keepaspectratio}
% Set default figure placement to htbp
\makeatletter
\def\fps@figure{htbp}
\makeatother
\setlength{\emergencystretch}{3em} % prevent overfull lines
\providecommand{\tightlist}{%
  \setlength{\itemsep}{0pt}\setlength{\parskip}{0pt}}
\setcounter{secnumdepth}{-\maxdimen} % remove section numbering
\ifLuaTeX
  \usepackage{selnolig}  % disable illegal ligatures
\fi
\IfFileExists{bookmark.sty}{\usepackage{bookmark}}{\usepackage{hyperref}}
\IfFileExists{xurl.sty}{\usepackage{xurl}}{} % add URL line breaks if available
\urlstyle{same} % disable monospaced font for URLs
\hypersetup{
  pdftitle={hw2},
  pdfauthor={Christopher Huong},
  hidelinks,
  pdfcreator={LaTeX via pandoc}}

\title{hw2}
\author{Christopher Huong}
\date{2024-01-25}

\begin{document}
\maketitle

\begin{enumerate}
\def\labelenumi{\Alph{enumi})}
\setcounter{enumi}{16}
\item
  2E1. Which of the expressions below correspond to the statement: the
  probability of rain on Monday?
\item
  \begin{enumerate}
  \def\labelenumii{(\arabic{enumii})}
  \tightlist
  \item
    P(rain\textbar Monday)
  \end{enumerate}
\end{enumerate}

\begin{center}\rule{0.5\linewidth}{0.5pt}\end{center}

\begin{enumerate}
\def\labelenumi{\Alph{enumi})}
\setcounter{enumi}{16}
\item
  2E2. Which of the following statements corresponds to the expression:
  P(Monday\textbar rain)?
\item
  \begin{enumerate}
  \def\labelenumii{(\arabic{enumii})}
  \setcounter{enumii}{2}
  \tightlist
  \item
    The probability that it is Monday, given it is raining
  \end{enumerate}
\end{enumerate}

\begin{center}\rule{0.5\linewidth}{0.5pt}\end{center}

\begin{enumerate}
\def\labelenumi{\Alph{enumi})}
\setcounter{enumi}{16}
\item
  2E3. Which of the expressions below correspond to the statement: the
  probability that it is Monday, given that it is raining?
\item
  \begin{enumerate}
  \def\labelenumii{(\arabic{enumii})}
  \tightlist
  \item
    P(Monday\textbar rain)
  \end{enumerate}
\end{enumerate}

\begin{center}\rule{0.5\linewidth}{0.5pt}\end{center}

\begin{enumerate}
\def\labelenumi{\Alph{enumi})}
\setcounter{enumi}{16}
\item
  2E4. \ldots. Discuss the globe tossing example from the chapter, in
  light of this statement. What does it mean to say ``the probability of
  water is 0.7''?
\item
  The observer's strength of belief that the next toss will result in
  water is 70\%.
\end{enumerate}

\begin{center}\rule{0.5\linewidth}{0.5pt}\end{center}

\begin{enumerate}
\def\labelenumi{\Alph{enumi})}
\setcounter{enumi}{16}
\tightlist
\item
  2M1. Recall the globe tossing model from the chapter. Compute and plot
  the grid approximate posterior distribution for each of the following
  sets of observations. In each case, assume a uniform prior for p.
\end{enumerate}

\begin{enumerate}
\def\labelenumi{(\arabic{enumi})}
\tightlist
\item
  W, W, W
\end{enumerate}

\begin{Shaded}
\begin{Highlighting}[]
\NormalTok{n}\OtherTok{=}\DecValTok{11} \CommentTok{\#10 sided globe}
\CommentTok{\#make list of parameter values on the grid}
\NormalTok{p\_grid }\OtherTok{\textless{}{-}} \FunctionTok{seq}\NormalTok{(}\AttributeTok{from=}\DecValTok{0}\NormalTok{, }\AttributeTok{to=}\DecValTok{1}\NormalTok{, }\AttributeTok{length.out=}\NormalTok{n)}
\CommentTok{\#uniform prior at each parameter value on grid}
\NormalTok{prior }\OtherTok{\textless{}{-}} \FunctionTok{rep}\NormalTok{(}\DecValTok{1}\NormalTok{,n)}
\CommentTok{\#compute likelihood at each parameter value}
\NormalTok{likelihood }\OtherTok{\textless{}{-}} \FunctionTok{dbinom}\NormalTok{(}\AttributeTok{x=}\DecValTok{3}\NormalTok{, }\AttributeTok{size=}\DecValTok{3}\NormalTok{, }\AttributeTok{prob=}\NormalTok{p\_grid)}
\NormalTok{unstd.posterior }\OtherTok{\textless{}{-}}\NormalTok{ prior }\SpecialCharTok{*}\NormalTok{ likelihood}
\NormalTok{posterior }\OtherTok{\textless{}{-}}\NormalTok{ unstd.posterior }\SpecialCharTok{/} \FunctionTok{sum}\NormalTok{(unstd.posterior)}
\FunctionTok{plot}\NormalTok{( p\_grid , posterior , }\AttributeTok{type=}\StringTok{"b"}\NormalTok{ )}
\end{Highlighting}
\end{Shaded}

\includegraphics{hw2_files/figure-latex/unnamed-chunk-1-1.pdf}

\begin{enumerate}
\def\labelenumi{(\arabic{enumi})}
\setcounter{enumi}{1}
\tightlist
\item
  W, W, W, L
\end{enumerate}

\begin{Shaded}
\begin{Highlighting}[]
\NormalTok{likelihood }\OtherTok{\textless{}{-}} \FunctionTok{dbinom}\NormalTok{(}\AttributeTok{x=}\DecValTok{3}\NormalTok{, }\AttributeTok{size=}\DecValTok{4}\NormalTok{, }\AttributeTok{prob=}\NormalTok{p\_grid)}
\NormalTok{unstd.posterior }\OtherTok{\textless{}{-}}\NormalTok{ prior }\SpecialCharTok{*}\NormalTok{ likelihood}
\NormalTok{posterior }\OtherTok{\textless{}{-}}\NormalTok{ unstd.posterior }\SpecialCharTok{/} \FunctionTok{sum}\NormalTok{(unstd.posterior)}
\FunctionTok{plot}\NormalTok{( p\_grid , posterior , }\AttributeTok{type=}\StringTok{"b"}\NormalTok{ )}
\end{Highlighting}
\end{Shaded}

\includegraphics{hw2_files/figure-latex/unnamed-chunk-2-1.pdf}

\begin{enumerate}
\def\labelenumi{(\arabic{enumi})}
\setcounter{enumi}{2}
\tightlist
\item
  L, W, W, L, W, W, W
\end{enumerate}

\begin{Shaded}
\begin{Highlighting}[]
\NormalTok{likelihood }\OtherTok{\textless{}{-}} \FunctionTok{dbinom}\NormalTok{(}\AttributeTok{x=}\DecValTok{5}\NormalTok{, }\AttributeTok{size=}\DecValTok{7}\NormalTok{, }\AttributeTok{prob=}\NormalTok{p\_grid)}
\NormalTok{unstd.posterior }\OtherTok{\textless{}{-}}\NormalTok{ prior }\SpecialCharTok{*}\NormalTok{ likelihood}
\NormalTok{posterior }\OtherTok{\textless{}{-}}\NormalTok{ unstd.posterior }\SpecialCharTok{/} \FunctionTok{sum}\NormalTok{(unstd.posterior)}
\FunctionTok{plot}\NormalTok{( p\_grid , posterior , }\AttributeTok{type=}\StringTok{"b"}\NormalTok{ )}
\end{Highlighting}
\end{Shaded}

\includegraphics{hw2_files/figure-latex/unnamed-chunk-3-1.pdf} ***

\begin{enumerate}
\def\labelenumi{\Alph{enumi})}
\setcounter{enumi}{16}
\tightlist
\item
  2M2. Now assume a prior for p that is equal to zero when p \textless{}
  0.5 and is a positive constant when p ≥ 0.5. Again compute and plot
  the grid approximate posterior distribution for each of the sets of
  observations in the problem just above.
\end{enumerate}

\begin{enumerate}
\def\labelenumi{(\arabic{enumi})}
\tightlist
\item
  W, W, W
\end{enumerate}

\begin{Shaded}
\begin{Highlighting}[]
\NormalTok{prior }\OtherTok{\textless{}{-}} \FunctionTok{ifelse}\NormalTok{(p\_grid }\SpecialCharTok{\textless{}} \FloatTok{0.5}\NormalTok{, }\DecValTok{0}\NormalTok{, }\DecValTok{1}\NormalTok{)}

\NormalTok{likelihood }\OtherTok{\textless{}{-}} \FunctionTok{dbinom}\NormalTok{(}\AttributeTok{x=}\DecValTok{3}\NormalTok{, }\AttributeTok{size=}\DecValTok{3}\NormalTok{, }\AttributeTok{prob=}\NormalTok{p\_grid)}
\NormalTok{unstd.posterior }\OtherTok{\textless{}{-}}\NormalTok{ prior }\SpecialCharTok{*}\NormalTok{ likelihood}
\NormalTok{posterior }\OtherTok{\textless{}{-}}\NormalTok{ unstd.posterior }\SpecialCharTok{/} \FunctionTok{sum}\NormalTok{(unstd.posterior)}
\FunctionTok{plot}\NormalTok{( p\_grid , posterior , }\AttributeTok{type=}\StringTok{"b"}\NormalTok{ )}
\end{Highlighting}
\end{Shaded}

\includegraphics{hw2_files/figure-latex/unnamed-chunk-4-1.pdf}

\begin{enumerate}
\def\labelenumi{(\arabic{enumi})}
\setcounter{enumi}{1}
\tightlist
\item
  W, W, W, L
\end{enumerate}

\begin{Shaded}
\begin{Highlighting}[]
\NormalTok{likelihood }\OtherTok{\textless{}{-}} \FunctionTok{dbinom}\NormalTok{(}\AttributeTok{x=}\DecValTok{3}\NormalTok{, }\AttributeTok{size=}\DecValTok{4}\NormalTok{, }\AttributeTok{prob=}\NormalTok{p\_grid)}
\NormalTok{unstd.posterior }\OtherTok{\textless{}{-}}\NormalTok{ prior }\SpecialCharTok{*}\NormalTok{ likelihood}
\NormalTok{posterior }\OtherTok{\textless{}{-}}\NormalTok{ unstd.posterior }\SpecialCharTok{/} \FunctionTok{sum}\NormalTok{(unstd.posterior)}
\FunctionTok{plot}\NormalTok{( p\_grid , posterior , }\AttributeTok{type=}\StringTok{"b"}\NormalTok{ )}
\end{Highlighting}
\end{Shaded}

\includegraphics{hw2_files/figure-latex/unnamed-chunk-5-1.pdf}

\begin{enumerate}
\def\labelenumi{(\arabic{enumi})}
\setcounter{enumi}{2}
\tightlist
\item
  L, W, W, L, W, W, W
\end{enumerate}

\begin{Shaded}
\begin{Highlighting}[]
\NormalTok{likelihood }\OtherTok{\textless{}{-}} \FunctionTok{dbinom}\NormalTok{(}\AttributeTok{x=}\DecValTok{5}\NormalTok{, }\AttributeTok{size=}\DecValTok{7}\NormalTok{, }\AttributeTok{prob=}\NormalTok{p\_grid)}
\NormalTok{unstd.posterior }\OtherTok{\textless{}{-}}\NormalTok{ prior }\SpecialCharTok{*}\NormalTok{ likelihood}
\NormalTok{posterior }\OtherTok{\textless{}{-}}\NormalTok{ unstd.posterior }\SpecialCharTok{/} \FunctionTok{sum}\NormalTok{(unstd.posterior)}
\FunctionTok{plot}\NormalTok{( p\_grid , posterior , }\AttributeTok{type=}\StringTok{"b"}\NormalTok{ )}
\end{Highlighting}
\end{Shaded}

\includegraphics{hw2_files/figure-latex/unnamed-chunk-6-1.pdf}

\begin{center}\rule{0.5\linewidth}{0.5pt}\end{center}

\begin{enumerate}
\def\labelenumi{\Alph{enumi})}
\setcounter{enumi}{16}
\item
  2M3. Suppose there are two globes, one for Earth and one for Mars. The
  Earth globe is 70\% covered in water. The Mars globe is 100\% land.
  Further suppose that one of these globes---you don't know which---was
  tossed in the air and produced a ``land'' observation. Assume that
  each globe was equally likely to be tossed. Show that the posterior
  probability that the globe was the Earth, conditional on seeing
  ``land'' (Pr(Earth\textbar land)), is 0.23.
\item
  Given: P(land\textbar Earth) = 0.3 P(land\textbar Mars) = 1.0 P(Earth)
  = 0.5 P(Mars) = 0.5 P(land) = P(land\textbar Earth) * P(Earth) +
  P(land\textbar Mars) * P(Mars)
\end{enumerate}

Bayes theorem: P(A\textbar B) = P(B\textbar A)\emph{P(A) / P(B)
P(Earth\textbar land) = (P(land\textbar Earth) } P(Earth)) / P(land)

\begin{Shaded}
\begin{Highlighting}[]
\NormalTok{p\_land }\OtherTok{=} \FloatTok{0.3} \SpecialCharTok{*} \FloatTok{0.5} \SpecialCharTok{+} \FloatTok{1.0} \SpecialCharTok{*} \FloatTok{0.5}
\NormalTok{p\_land}
\end{Highlighting}
\end{Shaded}

\begin{verbatim}
## [1] 0.65
\end{verbatim}

\begin{Shaded}
\begin{Highlighting}[]
\NormalTok{p\_earth\_given\_land }\OtherTok{=}\NormalTok{ (}\FloatTok{0.3} \SpecialCharTok{*} \FloatTok{0.5}\NormalTok{) }\SpecialCharTok{/} \FloatTok{0.65}
\NormalTok{p\_earth\_given\_land}
\end{Highlighting}
\end{Shaded}

\begin{verbatim}
## [1] 0.2307692
\end{verbatim}

\begin{center}\rule{0.5\linewidth}{0.5pt}\end{center}

\begin{enumerate}
\def\labelenumi{\Alph{enumi})}
\setcounter{enumi}{16}
\item
  2M4. Suppose you have a deck with only three cards. Each card has two
  sides, and each side is either black or white. One card has two black
  sides. The second card has one black and one white side. The third
  card has two white sides. Now suppose all three cards are placed in a
  bag and shuffled. Someone reaches into the bag and pulls out a card
  and places it flat on a table. A black side is shown facing up, but
  you don't know the color of the side facing down. Show that the
  probability that the other side is also black is 2/3. Use the counting
  method (Section 2 of the chapter) to approach this problem. This means
  counting up the ways that each card could produce the observed data (a
  black side facing up on the table).
\item
  Given: Card 1 = 2 black 0 white = 2 ways to produce the observed data
  Card 2 = 1 black 1 white = 1 way to produce the observed data Card 3 =
  0 black 2 white = 0 ways to produce the observed data
\end{enumerate}

Total 3 ways to produce the observed data P(Card1\textbar Black) = 2/3
P(Card2\textbar Black) = 1/3 P(Card3\textbar Black) = 0/3

Only Card1 will have the other side black, thus that probability is 2/3

\begin{center}\rule{0.5\linewidth}{0.5pt}\end{center}

\begin{enumerate}
\def\labelenumi{\Alph{enumi})}
\setcounter{enumi}{16}
\item
  2M5. Now suppose there are four cards: B/B, B/W, W/W, and another B/B.
  Again suppose a card is drawn from the bag and a black side appears
  face up. Again calculate the probability that the other side is black.
\item
  Given: Card 1 = 2 black 0 white = 2 ways to produce the observed data
  Card 2 = 1 black 1 white = 1 way to produce the observed data Card 3 =
  0 black 2 white = 0 ways to produce the observed data Card 4 = 1 black
  1 white = 2 ways to produce the observed data
\end{enumerate}

Total 5 ways to produce the observed data

P(Card1\textbar Black) = 2/5 P(Card2\textbar Black) = 1/5
P(Card3\textbar Black) = 0/5 P(Card4\textbar Black) = 2/5

Cards 1 and 4 will have the other side also black, thus that probability
is 4/5

\begin{center}\rule{0.5\linewidth}{0.5pt}\end{center}

\begin{enumerate}
\def\labelenumi{\Alph{enumi})}
\setcounter{enumi}{16}
\item
  2M6. Imagine that black ink is heavy, and so cards with black sides
  are heavier than cards with white sides. As a result, it's less likely
  that a card with black sides is pulled from the bag. So again assume
  there are three cards: B/B, B/W, and W/W. After experimenting a number
  of times, you conclude that for every way to pull the B/B card from
  the bag, there are 2 ways to pull the B/W card and 3 ways to pull the
  W/W card. Again suppose that a card is pulled and a black side appears
  face up. Show that the probability the other side is black is now 0.5.
  Use the counting method, as before
\item
  Given: Card 1 = 2 black 0 white = 2 ways to produce observed data Card
  2 = 1 black 1 white \emph{2 = 2 ways to produce observed data Card 3 =
  0 black 2 white } 3 = 0 ways to produce observed data
\end{enumerate}

Of the 4 ways to produce observed data (black side up), 2 of those also
have a black side down. 2/4 = 0.5

\begin{center}\rule{0.5\linewidth}{0.5pt}\end{center}

\begin{enumerate}
\def\labelenumi{\Alph{enumi})}
\setcounter{enumi}{16}
\item
  2M7. Assume again the original card problem, with a single card
  showing a black side face up. Before looking at the other side, we
  draw another card from the bag and lay it face up on the table. The
  face that is shown on the new card is white. Show that the probability
  that the first card, the one showing a black side, has black on its
  other side is now 0.75. Use the counting method, if you can. Hint:
  Treat this like the sequence of globe tosses, counting all the ways to
  see each observation, for each possible first card.
\item
  Give: Card 1 = 2 black 0 white Card 2 = 1 black 1 white Card 3 = 0
  black 2 white
\end{enumerate}

Possible draws to produce the observed data (black, then white) Card1
(B1), Card2 (W) Card1 (B2), Card2 (W) Card1 (B1), Card3 (W1) Card1 (B1),
Card3 (W2) Card1 (B2), Card3 (W1) Card1 (B2), Card3 (W2) Card2 (B),
Card3 (W1) Card2 (B), Card3 (W2)

\hypertarget{of-the-8-ways-to-produce-the-data-6-of-them-include-card-1-as-the-card-which-would-have-the-other-side-of-the-first-card-be-black-as-well.}{%
\subsection{Of the 8 ways to produce the data, 6 of them include Card 1
as the card, which would have the other side of the first card be black
as
well.}\label{of-the-8-ways-to-produce-the-data-6-of-them-include-card-1-as-the-card-which-would-have-the-other-side-of-the-first-card-be-black-as-well.}}

Thus that probability is 6/8 = 0.75

\begin{center}\rule{0.5\linewidth}{0.5pt}\end{center}

\begin{enumerate}
\def\labelenumi{\Alph{enumi})}
\setcounter{enumi}{16}
\item
  2H1. Suppose there are two species of panda bear. Both are equally
  common in the wild and live in the same places. They look exactly
  alike and eat the same food, and there is yet no genetic assay capable
  of telling them apart. They differ however in their family sizes.
  Species A gives birth to twins 10\% of the time, otherwise birthing a
  single infant. Species B births twins 20\% of the time, otherwise
  birthing singleton infants. Assume these numbers are known with
  certainty, from many years of field research. Now suppose you are
  managing a captive panda breeding program. You have a new female panda
  of unknown species, and she has just given birth to twins. What is the
  probability that her next birth will also be twins?
\item
  Given: P(PandaA) = 0.5 P(PandaB) = 0.5 P(Twins\textbar PandaA) = 0.10
  P(Twins\textbar PandaB) = 0.20 P(Twins) = 0.15
\end{enumerate}

Bayes theorem: P(A\textbar B) = P(B\textbar A)*P(A) / P(B)

Compute posterior probability of the panda being species A and B given
she just birthed twins

P(PandaA\textbar Twins) = P(Twins\textbar PandaA) * P(PandaA) / P(Twins)
= (0.10 * 0.5) / 0.15 = 0.333

P(PandaB\textbar Twins) = P(Twins\textbar PandaB) * P(PandaB) / P(Twins)
= (0.20 * 0.5) / 0.15 = 0.667

Compute updated probabilities of twins for each species, given twins
were just observed

\hypertarget{ptwins2-ptwinspandaa-ppandaa-ptwinspanda-b-ppandab}{%
\subsection{P(Twins2) = P(Twins\textbar PandaA) * P(PandaA) +
P(Twins\textbar Panda B) *
P(PandaB)}\label{ptwins2-ptwinspandaa-ppandaa-ptwinspanda-b-ppandab}}

\hypertarget{section}{%
\subsection{= 0.10 * 0.333 + 0.20 * 0.667}\label{section}}

\begin{center}\rule{0.5\linewidth}{0.5pt}\end{center}

\begin{enumerate}
\def\labelenumi{\Alph{enumi})}
\setcounter{enumi}{16}
\tightlist
\item
  2H2. Recall all the facts from the problem above. Now compute the
  probability that the panda we have is from species A, assuming we have
  observed only the first birth and that it was twins
\end{enumerate}

\hypertarget{a-ppandaatwins-ptwinspandaa-ppandaa-ptwins}{%
\subsection{A) P(PandaA\textbar Twins) = P(Twins\textbar PandaA) *
P(PandaA) / P(Twins)}\label{a-ppandaatwins-ptwinspandaa-ppandaa-ptwins}}

\hypertarget{section-1}{%
\subsection{= (0.10 * 0.5) / 0.15 = 0.333}\label{section-1}}

\begin{center}\rule{0.5\linewidth}{0.5pt}\end{center}

\begin{enumerate}
\def\labelenumi{\Alph{enumi})}
\setcounter{enumi}{16}
\item
  2H3. Continuing on from the previous problem, suppose the same panda
  mother has a second birth and that it is not twins, but a singleton
  infant. Compute the posterior probability that this panda is species A
\item
  P('Twins) = P('Twins\textbar PandaA) * P(PandaA) +
  P('Twins\textbar PandaB) * P(PandaB) = (1-.10) * (0.333) + (1-0.20) *
  (0.667) = 0.833
\end{enumerate}

\hypertarget{ppandaatwins-ptwinspandaa-ppandaa-ptwins}{%
\subsection{P(PandaA\textbar'Twins) = P('Twins\textbar PandaA) *
P(PandaA) / P('Twins)}\label{ppandaatwins-ptwinspandaa-ppandaa-ptwins}}

\hypertarget{section-2}{%
\subsection{= (0.90 * 0.333) / 0.833 = 0.360}\label{section-2}}

\begin{center}\rule{0.5\linewidth}{0.5pt}\end{center}

\begin{enumerate}
\def\labelenumi{\Alph{enumi})}
\setcounter{enumi}{16}
\item
  2H4. A common boast of Bayesian statisticians is that Bayesian
  inference makes it easy to use all of the data, even if the data are
  of different types. So suppose now that a veterinarian comes along who
  has a new genetic test that she claims can identify the species of our
  mother panda. But the test, like all tests, is imperfect. This is the
  information you have about the test: • The probability it correctly
  identifies a species A panda is 0.8. • The probability it correctly
  identifies a species B panda is 0.65. The vet administers the test to
  your panda and tells you that the test is positive for species A.
  First ignore your previous information from the births and compute the
  posterior probability that your panda is species A. Then redo your
  calculation, now using the birth data as well
\item
  Given: P(TestA\textbar PandaA) = 0.80 P(TestA\textbar PandaB) = 1-0.65
  = 0.35 P(TestA) = 0.80 * 0.50 + 0.35 * 0.50 = 0.575
\end{enumerate}

P(PandaA\textbar TestA) = P(TestA\textbar PandaA) * P(PandaA) / P(TestA)
= 0.80 * 0.50 / 0.575 = 0.696

\end{document}
